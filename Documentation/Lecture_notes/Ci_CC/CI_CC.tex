\documentclass[12pt]{article}
\usepackage{amsmath, amssymb}
\usepackage{physics}
\usepackage{geometry}
\geometry{margin=1in}
\usepackage{hyperref}

\title{Lecture Notes: Configuration Interaction and Coupled Cluster Theory}
\author{}
\date{}

\begin{document}

\maketitle

\section{Introduction}

Up to now we considered the use of small expansions (Configuration Interaction, CI) to calculate approximate:
\begin{itemize}
    \item Ionization potentials
    \item Electron affinities
    \item Excitation energies
\end{itemize}

These use the excitations of the Hartree--Fock (HF) ground-state determinant.

\[
E(\Psi) = \langle \Psi | H | \Psi \rangle
\]

Such expansions often give good \textit{total energies} (about 99\% of the correlation energy) but are \textbf{not size-consistent}.
On a chemical scale, large errors appear.
CI cannot be reliably used to calculate reaction energies or potential energy surfaces.

\section{Beyond CI}

We need to go beyond simple CI. The simplest approach is still the \textit{variational principle}.

\subsection{Single-Reference Approaches}

\begin{enumerate}
    \item \textbf{Configuration Interaction (CISD, CEPA)} \\
    Problems:
    \begin{itemize}
        \item Not size-consistent (\(E_{AB} \neq E_A + E_B\))
        \item Often insufficient for high accuracy
    \end{itemize}

    \item \textbf{Fixing Size-Consistency: Coupled Cluster (CC)} \\
    CCSD(T) is excellent for good HF references, but:
    \begin{itemize}
        \item Limited for bond breaking
        \item Poor convergence outside a good one-particle basis set
    \end{itemize}
\end{enumerate}

\subsection{Multi-Reference and Advanced Methods}

\begin{itemize}
    \item Solving the \textbf{multi-reference problem}:
    \begin{itemize}
        \item MR--CI
        \item MR--ACPF (size-consistent)
        \item MR--CC
    \end{itemize}

    \item \textbf{Excited States:}
    \begin{itemize}
        \item Equation-of-Motion Coupled Cluster (EOM-CC, CCSDT)
        \item Similarity-transformed Fock
    \end{itemize}

    \item \textbf{Large Systems:}
    \begin{itemize}
        \item Local correlation
        \item Explicit correlation (F12 methods)
    \end{itemize}
\end{itemize}

\section{Configuration Interaction (CI)}

We approximate the wavefunction as:
\[
\Psi = \Phi_0 + \sum_i c_i \Phi_i + \sum_{ij} c_{ij} \Phi_{ij} + \cdots
\]

\begin{itemize}
    \item CISD: singles + doubles excitations.
    \item Full CI: all possible \(N\)-fold excitations for \(N\) electrons.
\end{itemize}

\subsection{Hartree--Fock Singles and Brillouin Condition}

For a singly excited determinant:
\[
\langle \Phi_i^a | H | \Phi_{\mathrm{HF}} \rangle = F_{ai}
\]

In HF orbitals, the \textbf{Brillouin condition} applies:
\[
\langle \Phi_i^a | H | \Phi_{\mathrm{HF}} \rangle = 0
\]

Thus the HF energy cannot be lowered by adding only single excitations.

\section{Limitations of CI}

\subsection{Size Consistency}

CISD and higher single-reference CI are \textbf{not size-consistent}:
\[
E_{AB} \neq E_A + E_B
\]

This violates the extensivity principle and makes CI unreliable for separated subsystems.

\section{Coupled Cluster (CC) Theory}

The CC wavefunction is defined as:
\[
\Psi = e^T \Phi_0
\]

where
\[
T = T_1 + T_2 + T_3 + \cdots
\]

Each \(T_n\) represents all \(n\)-tuple excitations.
This exponential ansatz ensures \textbf{size-consistency} and good scaling.

\subsection{CC Equations}

Define the similarity-transformed Hamiltonian:
\[
\bar{H} = e^{-T} H e^T
\]

Project onto the reference and excited determinants to obtain the amplitude equations.
Note: \(\bar{H}\) is \textbf{non-Hermitian}.

\subsection{Expansion of \(\bar{H}\)}

\[
\bar{H} = H + [H, T] + \tfrac{1}{2} [[H, T], T] + \cdots
\]

This nested commutator expansion \textit{terminates} because \(T\) only contains excitations.

\section{Practical CC Levels}

\begin{itemize}
    \item \textbf{CCSD}: singles and doubles
    \item \textbf{CCSDT}: singles, doubles, triples (very expensive)
    \item \textbf{CCSD(T)}: CCSD with perturbative triples — the \textbf{gold standard} of quantum chemistry
\end{itemize}

CC methods work well for:
\begin{itemize}
    \item Geometries
    \item Vibrational frequencies
    \item Thermochemistry
\end{itemize}

\noindent
But can fail for \textbf{multi-reference} cases (e.g., bond breaking).

\section{Excited States}

\begin{itemize}
    \item \textbf{EOM-CC}: Equation-of-Motion CC for excited states, ionization potentials (IP-CC), electron affinities (EA-CC).
    \item Spin-flip and multi-reference CC exist for strongly correlated systems.
\end{itemize}

\section{Accuracy}

\begin{itemize}
    \item CCSD(T) typically yields \(\sim 0.01\)--\(0.02\ \mathrm{eV}\) accuracy for valence states.
    \item Not reliable for Rydberg states.
    \item EOM-CC IP/EA works well for small molecules.
\end{itemize}

\section{Summary}

\begin{itemize}
    \item CI is conceptually simple but suffers from size-consistency and scaling issues.
    \item Coupled Cluster (CC) uses an exponential ansatz ensuring size-consistency.
    \item CCSD(T) is the practical ``gold standard'' for many systems.
    \item EOM-CC extends CC to excited, ionized, and electron-attached states.
    \item Multi-reference situations (e.g., bond breaking) require more advanced MR-CC methods.
\end{itemize}

\end{document}
