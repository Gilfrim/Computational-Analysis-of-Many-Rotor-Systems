\documentclass[a4paper,10pt]{article}
\usepackage[a4paper,
    left=2cm,
    right=2cm,
    top=1.5cm,
    bottom=1.5cm]{geometry}

% Encoding and language
\usepackage[utf8]{inputenc}   % Handle UTF-8 characters
\usepackage[T1]{fontenc}      % Better font encoding
\usepackage[english]{babel}   % Language support

\setlength{\parindent}{0pt}

% Math and symbols
\usepackage{amsmath, amssymb}

% Graphics
\usepackage{graphicx}

% Hyperlinks
\usepackage[colorlinks=true, linkcolor=blue, urlcolor=blue, citecolor=blue]{hyperref}

% Bibliography (comment out if not using yet)
%\usepackage[backend=biber,style=authoryear]{biblatex}
%\addbibresource{refs.bib}

\usepackage[utf8]{inputenc}
\usepackage[T1]{fontenc}
\usepackage[english]{babel}
\usepackage{forest}

\begin{document}

\section*{Code Documentation}

This document is created to give a general overview of the code for the means of reviewing
and tracking progress as well as for training purposes.
\newline 
\newline The GitHub repository contains more files that could be useful, but
this document would only talk about the "quant\_rotor" library. The files in a discarded folder are either completed projects or discarded ideas; files that are in the main fouler are usually files with the work in progress.

\section{Library File Structure and Breakdown}

This section will provide a general overview for the library structure as well as a small note on every
file and a reference to more material about those files.\newline

The library is broken down into three main folders:
\newline \newline
- "core" contains the main files that are to be run or imported. They have the most high-level functions.

- "data" was designed to hold any input of output of the code it is not in use at the moment.

- "models" holds all supporting files and code which are used by "core".
\newline

\begin{forest}
for tree={
    font=\ttfamily,
    grow'=0,
    child anchor=west,
    parent anchor=south,
    anchor=west,
    calign=first,
    inner xsep=7pt,
    edge path={
        \noexpand\path [draw, \forestoption{edge}]
        (!u.parent anchor) -- +(5pt,0) |- (.child anchor)\forestoption{edge label};
    },
    before typesetting nodes={
        if n=1
            {insert before={[,phantom]}}
            {}
    },
    fit=band,
    before computing xy={l=45pt},
}
  [quant\_rotor
    [core]
    [data]
    [models]
  ]
\end{forest}

Looking more specifically at the files They are marked with letters in parentheses as such:
\newline \newline
UD - for under development

F - finished file

FO - finished files gotten from external sources
\newline \newline
Additionally "core" and "models" are separated on two sub-folders which generally contain the same files but optimized for sparse or dense matrix 
handling.(consult Wikipedia for sparse matrices and sparse in python Scipy)

\begin{forest}
for tree={
    font=\ttfamily,
    grow'=0,
    child anchor=west,
    parent anchor=south,
    anchor=west,
    calign=first,
    inner xsep=7pt,
    edge path={
        \noexpand\path [draw, \forestoption{edge}]
        (!u.parent anchor) -- +(5pt,0) |- (.child anchor)\forestoption{edge label};
    },
    before typesetting nodes={
        if n=1
            {insert before={[,phantom]}}
            {}
    },
    fit=band,
    before computing xy={l=45pt},
}
  [core
    [dense
      [de\_solve\_one\_thermal\_dense.py (UD)]
      [de\_solve\_one\_thermal.py (F)]
      [de\_solve.py (FO)]
      [hamiltonian\_big.py (F)]
      [hamiltonian.py (F)]
      [t\_amplitudes\_periodic\_fast.py (UD)]
      [t\_amplitudes\_periodic.py (F)]
      [t\_amplitudes\_guess.py (F)]
    ]
    [sparse
      [de\_solve\_one\_thermal\_sparse.py (UD)]
      [hamiltonian\_big.py (F)]
      [hamiltonian.py (F)]
      [t\_amplitudes\_periodic\_fast.py (UD)]
    ]
  ]
\end{forest}

\begin{forest}
for tree={
    font=\ttfamily,
    grow'=0,
    child anchor=west,
    parent anchor=south,
    anchor=west,
    calign=first,
    inner xsep=7pt,
    edge path={
        \noexpand\path [draw, \forestoption{edge}]
        (!u.parent anchor) -- +(5pt,0) |- (.child anchor)\forestoption{edge label};
    },
    before typesetting nodes={
        if n=1
            {insert before={[,phantom]}}
            {}
    },
    fit=band,
    before computing xy={l=45pt},
}
  [models
    [dense
      [de\_solver\_func.py (FO)]
      [density\_matrix (F)]
      [stat\_mech\_thermo.py (F)]
      [support\_ham.py (F)]
      [t\_amplitudes\_sub\_class\_fast.py (UD)]
      [t\_amplitudes\_sub\_class.py (F)]
    ]
    [sparse
      [support\_ham.py (F)]
      [t\_amplitudes\_sub\_class\_fast.py (UD)]
    ]
  ]
\end{forest}

\section{File descriptions}

This section provides a short description of relevant files and their function structure from "core" with referencing to all files used from "models"; reference to any useful documentation on the topic. 

\subsection[short]{Classic approach of constructing a Hamiltonian: "hamiltonian" (dense), \& "hamiltonian" (sparse)}

Classical approach to constructing a Hamiltonian. (consult Configurational Coupled Cluster document Section 1 and 2)

The files are updated and optimized versions of each other in the order of newer to older: "hamiltonian" (sparse) -> "hamiltonian" (dense).


\textbf{\newline Structure of dense hamiltonian: \newline}

\begin{forest}
for tree={
    font=\ttfamily,
    grow'=0,
    child anchor=west,
    parent anchor=south,
    anchor=west,
    calign=first,
    inner xsep=7pt,
    edge path={
        \noexpand\path [draw, \forestoption{edge}]
        (!u.parent anchor) -- +(5pt,0) |- (.child anchor)\forestoption{edge label};
    },
    before typesetting nodes={
        if n=1
            {insert before={[,phantom]}}
            {}
    },
    fit=band,
    before computing xy={l=60pt},
}
  [...hamiltonian
    [quant\_rotor.models.dense.support\_ham]
  ]
\end{forest}

\textbf{\newline Structure of sparse hamiltonian: \newline}

\begin{forest}
for tree={
    font=\ttfamily,
    grow'=0,
    child anchor=west,
    parent anchor=south,
    anchor=west,
    calign=first,
    inner xsep=7pt,
    edge path={
        \noexpand\path [draw, \forestoption{edge}]
        (!u.parent anchor) -- +(5pt,0) |- (.child anchor)\forestoption{edge label};
    },
    before typesetting nodes={
        if n=1
            {insert before={[,phantom]}}
            {}
    },
    fit=band,
    before computing xy={l=60pt},
}
  [...hamiltonian
    [quant\_rotor.models.sparse.support\_ham]
  ]
\end{forest}

\subsection[short]{Reduced density matrix from ED approach: "hamiltonian\_big.py" (dense) \& "hamiltonian\_big.py" (sparse) }

Approach to constructing a hamiltonian by approximation through reduced density matrices. (consult Configurational Coupled Cluster document Section 1 and 2)

The files are updated and optimized versions of each other in the order of newer to older: "hamiltonian\_big" (sparse) -> "hamiltonian\_big" (dense).

\textbf{\newline Structure of dense hamiltonian\_big: \newline}

\begin{forest}
for tree={
    font=\ttfamily,
    grow'=0,
    child anchor=west,
    parent anchor=south,
    anchor=west,
    calign=first,
    inner xsep=7pt,
    edge path={
        \noexpand\path [draw, \forestoption{edge}]
        (!u.parent anchor) -- +(5pt,0) |- (.child anchor)\forestoption{edge label};
    },
    before typesetting nodes={
        if n=1
            {insert before={[,phantom]}}
            {}
    },
    fit=band,
    before computing xy={l=115pt},
}
  [...hamiltonian\_big
    [quant\_rotor.models.dense.density\_matrix]
    [quant\_rotor.core.dense.hamiltonian
      [quant\_rotor.models.dense.support\_ham]
    ]
  ]
\end{forest}

\textbf{\newline \newline Structure of sparse hamiltonian\_big: \newline}

\begin{forest}
for tree={
    font=\ttfamily,
    grow'=0,
    child anchor=west,
    parent anchor=south,
    anchor=west,
    calign=first,
    inner xsep=7pt,
    edge path={
        \noexpand\path [draw, \forestoption{edge}]
        (!u.parent anchor) -- +(5pt,0) |- (.child anchor)\forestoption{edge label};
    },
    before typesetting nodes={
        if n=1
            {insert before={[,phantom]}}
            {}
    },
    fit=band,
    before computing xy={l=115pt},
}
  [...hamiltonian\_big
    [quant\_rotor.models.sparse.density\_matrix]
    [quant\_rotor.core.sparse.hamiltonian
      [quant\_rotor.models.sparse.support\_ham]
    ]
  ]
\end{forest}

\subsection[short]{Iterative procedure to solve for residuals: "t\_amplitudes\_periodic\_fast" (dense and sparse) \&
"t\_amplitudes\_periodic" \& "t\_amplitudes\_guess"}

This files provides iterative approach to CCC.(consult Configurational Coupled Cluster document Sections 3, 5, 6 and 7) The files are
the updated and optimized versions of each other in the order of newer to older: "t\_amplitudes\_periodic\_fast" (dense) -> "t\_amplitudes\_periodic\_fast" (dense) -> "t\_amplitudes\_periodic".
\newline \newline
The "t\_amplitudes\_guess" file is used to make a prediction which than later be given to the iterative solver. Was introduce in attempt to 
medigate the problem of bit g. (consult Configurational Coupled Cluster document Section 8)

\textbf{\newline Structure of t\_amplitudes\_guess: \newline}

\begin{forest}
for tree={
    font=\ttfamily,
    grow'=0,
    child anchor=west,
    parent anchor=south,
    anchor=west,
    calign=first,
    inner xsep=7pt,
    edge path={
        \noexpand\path [draw, \forestoption{edge}]
        (!u.parent anchor) -- +(5pt,0) |- (.child anchor)\forestoption{edge label};
    },
    before typesetting nodes={
        if n=1
            {insert before={[,phantom]}}
            {}
    },
    fit=band,
    before computing xy={l=95pt},
}
  [...t\_amplitudes\_guess
    [quant\_rotor.models.dense.support\_ham]
  ]
\end{forest}

\textbf{\newline Structure of "t\_amplitudes\_periodic: \newline}

\begin{forest}
for tree={
    font=\ttfamily,
    grow'=0,
    child anchor=west,
    parent anchor=south,
    anchor=west,
    calign=first,
    inner xsep=7pt,
    edge path={
        \noexpand\path [draw, \forestoption{edge}]
        (!u.parent anchor) -- +(5pt,0) |- (.child anchor)\forestoption{edge label};
    },
    before typesetting nodes={
        if n=1
            {insert before={[,phantom]}}
            {}
    },
    fit=band,
    before computing xy={l=95pt},
}
  [..."t\_amplitudes\_periodic
    [quant\_rotor.models.dense.t\_amplitudes\_sub\_class]
    [quant\_rotor.models.dense.support\_ham]
  ]
\end{forest}

\textbf{\newline Structure of dense "t\_amplitudes\_periodic\_fast: \newline}

\begin{forest}
for tree={
    font=\ttfamily,
    grow'=0,
    child anchor=west,
    parent anchor=south,
    anchor=west,
    calign=first,
    inner xsep=7pt,
    edge path={
        \noexpand\path [draw, \forestoption{edge}]
        (!u.parent anchor) -- +(5pt,0) |- (.child anchor)\forestoption{edge label};
    },
    before typesetting nodes={
        if n=1
            {insert before={[,phantom]}}
            {}
    },
    fit=band,
    before computing xy={l=95pt},
}
  [..."t\_amplitudes\_periodic
    [quant\_rotor.models.dense.t\_amplitudes\_sub\_class\_fast]
    [quant\_rotor.models.dense.support\_ham]
  ]
\end{forest}

\textbf{\newline Structure of sparse "t\_amplitudes\_periodic\_fast: \newline}

\begin{forest}
for tree={
    font=\ttfamily,
    grow'=0,
    child anchor=west,
    parent anchor=south,
    anchor=west,
    calign=first,
    inner xsep=7pt,
    edge path={
        \noexpand\path [draw, \forestoption{edge}]
        (!u.parent anchor) -- +(5pt,0) |- (.child anchor)\forestoption{edge label};
    },
    before typesetting nodes={
        if n=1
            {insert before={[,phantom]}}
            {}
    },
    fit=band,
    before computing xy={l=95pt},
}
  [..."t\_amplitudes\_periodic
    [quant\_rotor.models.sparse.t\_amplitudes\_sub\_class\_fast]
    [quant\_rotor.models.sparse.support\_ham]
  ]
\end{forest}

\newpage
\subsection[short]{Time-dependent CCC approach: "de\_solve\_one\_thermal\_dense" \newline \& "de\_solve\_one\_thermal" \& "de\_solve" \& de\_solve\_one\_thermal\_sparse}

This files provides a Runge Kutta method for solving a differential equation to find the first and second residuals by time probagation.(consult Configurational Coubled Cluster document
Section 10) The files are updated and optimised versions of each other in the order of newer to older: "de\_solve\_one\_thermal\_sparse" -> "de\_solve\_one\_thermal\_dense" -> "de\_solve\_one\_thermal" -> "de\_solve".

\textbf{\newline Structure of de\_solve\_one\_thermal \& de\_solve: \newline}

\begin{forest}
for tree={
    font=\ttfamily,
    grow'=0,
    child anchor=west,
    parent anchor=south,
    anchor=west,
    calign=first,
    inner xsep=7pt,
    edge path={
        \noexpand\path [draw, \forestoption{edge}]
        (!u.parent anchor) -- +(5pt,0) |- (.child anchor)\forestoption{edge label};
    },
    before typesetting nodes={
        if n=1
            {insert before={[,phantom]}}
            {}
    },
    fit=band,
    before computing xy={l=125pt},
}
  [...de\_solve\_one\_thermal\_dense \& de\_solve
    [quant\_rotor.models.dense.de\_solver\_func]
    [quant\_rotor.models.dense.t\_amplitudes\_sub\_class]
    [quant\_rotor.models.dense.support\_ham]
  ]
\end{forest}

\textbf{\newline Structure of de\_solve\_one\_thermal\_dense: \newline}

\begin{forest}
for tree={
    font=\ttfamily,
    grow'=0,
    child anchor=west,
    parent anchor=south,
    anchor=west,
    calign=first,
    inner xsep=7pt,
    edge path={
        \noexpand\path [draw, \forestoption{edge}]
        (!u.parent anchor) -- +(5pt,0) |- (.child anchor)\forestoption{edge label};
    },
    before typesetting nodes={
        if n=1
            {insert before={[,phantom]}}
            {}
    },
    fit=band,
    before computing xy={l=95pt},
}
  [...de\_solve\_one\_thermal\_dense
    [quant\_rotor.models.dense.de\_solver\_func]
    [quant\_rotor.models.dense.t\_amplitudes\_sub\_class\_fast]
    [quant\_rotor.models.sparse.support\_ham]
  ]
\end{forest}

\textbf{\newline Structure of de\_solve\_one\_thermal\_sparse: \newline}

\begin{forest}
for tree={
    font=\ttfamily,
    grow'=0,
    child anchor=west,
    parent anchor=south,
    anchor=west,
    calign=first,
    inner xsep=7pt,
    edge path={
        \noexpand\path [draw, \forestoption{edge}]
        (!u.parent anchor) -- +(5pt,0) |- (.child anchor)\forestoption{edge label};
    },
    before typesetting nodes={
        if n=1
            {insert before={[,phantom]}}
            {}
    },
    fit=band,
    before computing xy={l=95pt},
}
  [...de\_solve\_one\_thermal\_sparse
    [quant\_rotor.models.dense.de\_solver\_func]
    [quant\_rotor.models.sparse.t\_amplitudes\_sub\_class\_fast]
    [quant\_rotor.models.sparse.support\_ham]
  ]
\end{forest}

\end{document}