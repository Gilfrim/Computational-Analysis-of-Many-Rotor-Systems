\documentclass[a4paper,5pt]{article}
\usepackage[a4paper,
    left=2cm,
    right=2cm,
    top=1.5cm,
    bottom=1.5cm]{geometry}

% Encoding and language
\usepackage[utf8]{inputenc}   % Handle UTF-8 characters
\usepackage[T1]{fontenc}      % Better font encoding
\usepackage[english]{babel}   % Language support

% Math and symbols
\usepackage{amsmath, amssymb}

% Graphics
\usepackage{graphicx}

% Hyperlinks
\usepackage[colorlinks=true, linkcolor=blue, urlcolor=blue, citecolor=blue]{hyperref}

% Bibliography (comment out if not using yet)
%\usepackage[backend=biber,style=authoryear]{biblatex}
%\addbibresource{refs.bib}

\usepackage[utf8]{inputenc}
\usepackage[T1]{fontenc}
\usepackage[english]{babel}
\usepackage{forest}

\begin{document}

\section{Code Documentation}

This document is created to give a general overview of the code for the means of reviewing seeing
and tracking progress as well as for traning purpuses.
\newline 
\newline The github repository contains more files which could be usefull but 
this document would only be talking about the "quant\_rotor" library. The files in a discarded foulder
are ether done projects or discarded ideas; files that are in the main foulder are ususally files with the work in 
progress.

\subsection[short]{Library File Structure and Breakdown}

\begin{forest}
for tree={
    font=\ttfamily,
    grow'=0,
    child anchor=west,
    parent anchor=south,
    anchor=west,
    calign=first,
    inner xsep=7pt,
    edge path={
        \noexpand\path [draw, \forestoption{edge}]
        (!u.parent anchor) -- +(5pt,0) |- (.child anchor)\forestoption{edge label};
    },
    before typesetting nodes={
        if n=1
            {insert before={[,phantom]}}
            {}
    },
    fit=band,
    before computing xy={l=45pt},
}
  [quant\_rotor
    [core]
    [data]
    [models]
  ]
\end{forest}

\begin{forest}
for tree={
    font=\ttfamily,
    grow'=0,
    child anchor=west,
    parent anchor=south,
    anchor=west,
    calign=first,
    inner xsep=7pt,
    edge path={
        \noexpand\path [draw, \forestoption{edge}]
        (!u.parent anchor) -- +(5pt,0) |- (.child anchor)\forestoption{edge label};
    },
    before typesetting nodes={
        if n=1
            {insert before={[,phantom]}}
            {}
    },
    fit=band,
    before computing xy={l=45pt},
}
  [core
    [dense
      [de\_solve\_one\_thermal\_dense.py]
      [de\_solve\_one\_thermal.py]
      [de\_solve.py]
      [hamiltonian\_big.py]
      [hamiltonian.py]
      [t\_amplitudes\_periodic\_new.py]
      [t\_amplitudes\_periodic.py]
    ]
    [sparse
      [de\_solve\_one\_thermal\_sparse.py]
      [de\_solve.py]
      [hamiltonian\_big.py]
      [hamiltonian.py]
      [t\_amplitudes\_periodic\_new\_check.py]
    ]
  ]
\end{forest}

\begin{forest}
for tree={
    font=\ttfamily,
    grow'=0,
    child anchor=west,
    parent anchor=south,
    anchor=west,
    calign=first,
    inner xsep=7pt,
    edge path={
        \noexpand\path [draw, \forestoption{edge}]
        (!u.parent anchor) -- +(5pt,0) |- (.child anchor)\forestoption{edge label};
    },
    before typesetting nodes={
        if n=1
            {insert before={[,phantom]}}
            {}
    },
    fit=band,
    before computing xy={l=45pt},
}
  [models
    [dense
      [de\_solver\_func.py]
      [density\_matrix]
      [stat\_mech\_thermo.py]
      [support\_ham.py]
      [t\_amplitudes\_guess.py]
      [t\_amplitudes\_sub\_class\_new.py]
      [t\_amplitudes\_sub\_class.py]
    ]
    [sparse
      [support\_ham.py]
      [model1.py]
      [model1.py]
      [model1.py]
      [model1.py]
      [model1.py]
      [model1.py]
      [model2.py]
    ]
  ]
\end{forest}

\section{Math Example}
Here is an inline equation: \( E = mc^2 \).  

And here is a displayed equation:
\[
    \int_0^\infty e^{-x^2} \, dx = \frac{\sqrt{\pi}}{2}.
\]

\section{Including a Figure}
\begin{figure}[h]
    \centering
    \includegraphics[width=0.5\textwidth]{example-image} % replace with your own file
    \caption{An example figure.}
    \label{fig:example}
\end{figure}

\section{Conclusion}
That’s all for now. Happy \LaTeX{} writing!

% Uncomment if you use biblatex
%\printbibliography

\end{document}